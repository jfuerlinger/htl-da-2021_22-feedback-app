Aufgrund von diversen Schwierigkeiten, die wir während der Diplomarbeit konfrontiert waren, konnten einige 
Funktionen der App nicht rechtzeitig implementiert werden.

\section{Feedback Kategorien}
\author{Stefano Pyringer}
Der Lehrende hätte bei der Erstellung bei der Lehreinheit, eigene Kategorien mit Sternen oder Kommentar anlegen können, 
um ein präziseres Feedback von den Schülern erhalten zu können. 

\section{Export als PDF}
\author{Stefano Pyringer}
Diese Funktion sollte den Ausdruck und externe Speicherung von Statistiken, Lehreinheiten und deren Bewertungen vereinfachen.

\section{digitale Erfolge/Abzeichen}
\author{Stefano Pyringer}
Bei dieser Idee wurde die PC-Spiele Plattform Steam als Vorbild genommen, die gewisse Erfolge mittels Abzeichen die im Benutzerprofil öffentlich
sichtbar gemacht werden können, um die Motivation für gewisse nicht verpflichtende Tätigkeiten im Spiel zu steigern. 
Dies sollte in der Feedback App ebenfalls möglich gemacht werden. Zudem kann der Lehrende auch sehen, ob der Bewerter aufgrund bestimmter Erfolgsabzeichen eventuell mehr Erfahrung 
mit dem Feedback geben hat und somit die Bewertung ehrlicher und eine mehr konstruierte Rückmeldung gegenüber anderen hat.

\section{Benachrichtigungen}
\author{Stefano Pyringer}
Auf den mobilen Endgerät sollte der Benutzer mittels benachrichtigt werden, wenn er Feedback von anderen Benutzern erhalten hat oder seine 
Lehreinheit das Ablaufdatum erreicht hat. Zudem war eine Benachrichtigung mittel E-Mail geplant.

\section{Two-Faktor Authentifizierung}
\author{Stefano Pyringer}
Der Benutzer hätte entscheiden können, ob er die sichere 2-Faktor Anmeldemethode verwendet. Dabei handelt es sich um eine 
Eingabe eines zweiten Kennworts, dass automatisch von einem externen Dienst wie Google Authenticator erstellt wurde. 
Dieser Pin ist zeitlich begrenzt und erhöht damit erheblich die Sicherheit, um den Missbrauch eines Accounts zu verhindern.

\section{Benutzerkonto Profilfoto}
\author{Stefano Pyringer}
Die Hinzufügung eines Profilbilds hätte der User sein Benutzerkonto individueller gestalten können. Diese Bild wäre in einem 
platzsparenden Format (JPEG) mit Höhen- und Längenbegrenzung auf dem Server mit der Feedback Datenbank in der Tabelle User hochgeladen worden.
