\section{Backend}
\author{Stefano Pyringer}

\subsection{Datenbank}
\author{Stefano Pyringer}

\subsection{Web API}
\author{Stefano Pyringer}

\subsubsection{Registrierung und Login}
\author{Stefano Pyringer}
Für das Usermanagement wird das NuGet Package ASP NET Identity verwendet. Der Authenticate-Controller ermöglicht die Registrierung, den Login und die Löschung des Benutzerkontos. 
Zusätzlich kann das Passwort und die E-Mail Adresse nachträglich geändert werden.  
Bei erfolgreicher Authentifizierung wird ein gültiger JSON Web Token zurückgesendet. 

\begin{figure}[h]
    \includegraphics*[width=15cm]{./pics/Screenshot_Swagger_Auth.png}
    \caption[Swagger Authenticate-Controller]{Screenshot Swagger Web API Dokumentation Authenticate-Controller}
\end{figure}

Der Benutzername muss folgende Kriterien erfüllen:
\begin{itemize}
    \item mindestens 6 Zeichen
    \item maximal 26 Zeichen
    \item keine Leerzeichen
    \item keine Sonderzeichen (Umlaute sind erlaubt)
\end{itemize}

Das Passwort hat folgende Mindestanforderungen:
\begin{itemize}
    \item mindestens 6 Zeichen
    \item keine Leerzeichen
    \item mindestens einen Großbuchstaben, einen Kleinbuchstaben und eine Zahl
\end{itemize}

Diese Anforderungen sind in der Start-Up Konfiguration der ASP NET Core Web API festgelegt und werden zusätzlich von 
der Klasse AuthenticateValidations.cs kontrolliert. Im Fehlerfall wird ein entsprechender HTTP-Fehlercode zurückgegeben.

\begin{figure}[h]
    \includegraphics*[width=15cm]{./pics/screenshot_Startup_PwUserReq.png}
    \caption[PW User Requirements Startup]{Screenshot Startup.cs Benutzername und Passwort Anforderungen}
\end{figure}

\section{Frontend}
\author{Mirzet Sakonjic}
\subsection*{Login}
Die Anwendungsseite wird ausschließlich in der Oberfläche realisiert, 
indem das Bild in einem separaten Ressourcenordner gespeichert und mit <img> 
angezeigt wird. Das Bild auf der Startseite ist urheberrechtlich geschützt, 
da sich dieses eine Bild nur auf der Anmeldeseite befindet. Beim Einloggen gibt 
der Nutzer seine E-Mail-Adresse und sein Passwort ein, die bereits ausgelesen 
und im Hintergrund in der Datenbank gespeichert wurden. Bei Bedarf kann ein 
neuer Benutzer angelegt werden. Die Validierung überprüft, ob die E-Mail-Adresse 
und das Passwort eingegeben wurden. Es überprüft auch Passwortregeln wie Länge, 
Großbuchstaben oder zusätzliche Zeichen.
Wenn der User auf den Button „Login“ drückt, wird ein API Get-Request mit der
eingegebenen E-Mail und Passwort ans Backend gesendet. Wenn die Zugangsdaten
stimmen und der User vorhanden ist, werden die Logindaten inklusive Username
zurückgesendet und in der globalen Komponente abgespeichert, damit im gesamten
Projekt darauf zugegriffen werden kann.
Sollten die Zugangsdaten nicht stimmen, wird eine Fehlermeldung angezeigt.
Nach dem erfolgreichen Laden aller Daten wird der User zu der Startseite
weitergeleitet.

\subsection*{Benutzerkontoverwaltung}