\section{Ausgangslage}
\setauthor{Stefano Pyringer}


\section{Zielsetzung}
\setauthor{Stefano Pyringer}


\section{Theoretischer Hintergrund}
\setauthor{Stefano Pyringer}

\subsection{Was ist Feedback?}



\subsection{Johari-Fenster}
Die Persönlichkeitsmerkmale können unterschiedlich von Menschen oder von einem selbst wahrgenommen werden.
Das Johari-Fenster stellt die Unterschiede zwischen Selbst- und Fremdwahrnehmung dar. 
Es wurde 1955 von den amerikanischen Sozialpsychologen Joseph Luft und Harry Ingham entwickelt. Mithilfe des 
Feedback soll der öffentliche Bereich der Person erweitert werden.

\begin{figure}[h]
    \begin{center}
        \includegraphics*[width=8cm]{./pics/Johari.png}
        \caption[Johari-Fenster]{Johari-Fenster \cite{Johari}}
    \end{center}
\end{figure}

\subsubsection{Öffentliche Person}
In diesem Feld versammeln sich Merkmale, die der Person selbst und anderen Personen bekannt sind. 
Das Handeln der Person ist frei und unbeeinträchtigt von Ängsten und Vorbehalten.

\subsubsection{Offenkundige Person - blinder Fleck}
Hier finden sich Persönlichkeitsmerkmale, die der eigenen Person kaum bewusst ist, jedoch von anderen deutlich 
wahrgenommen werden können. Diese Merkmale sind sichtbar durch unbewusste Gewohnheiten und Verhaltensweisen, Vorurteile und 
persönliche Zu- und Abneigungen.

\subsubsection{Verborgene Person}
Diese Persönlichkeitsmerkmale werden von der Person bewusst verborgen sind nur dieser bekannt. Dies können 
heimliche Wünsche oder wunde Punkte sein.

\subsubsection{Unbewusste Person}
In diesem Feld sind die Merkmale der Person sich selbst und anderen Personen verborgen. Beispiel dafür sind 
verborgene Talente oder ungenützte Begabungen.

\section{Geplantes Ergebnis}
\setauthor{Stefano Pyringer}


\section{Aufbau der Diplomarbeit}
\setauthor{Stefano Pyringer}

\subsubsection{2. Systemarchitektur}


\subsubsection{3. Technologien}


\subsubsection{4. Entwicklungsumgebungen}


\subsubsection{5. Produktübersicht}


\subsubsection{6. Umsetzung}


\subsubsection{7. Nicht realisierte Funktionen}