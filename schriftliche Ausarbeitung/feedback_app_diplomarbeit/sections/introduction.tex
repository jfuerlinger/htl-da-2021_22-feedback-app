\section{Ausgangslage}
\setauthor{Stefano Pyringer}
Feedback ist eine Möglichkeit für Personen und Unternehmen, sich und ihre Produkte und Dienstleistungen zu verbessern. 
Besonders im Zeitalter des Internets ist eine Bewertung einfach und schnell möglich. 
Erfolgereiche Unternehmen verwenden Rezensionssysteme, um ihre Dienstleistungen nachhaltig zu verbessern. 

Aufbauend auf eigene Erfahrungen gab vor diesem Projekt keine zufriedenstellende Möglichkeit, ein sofortiges ausführliches Feedback zu geben, 
da die Lehrenden oft sofort in die nächste Unterrichtseinheit wechseln. Zudem tauchten Probleme öfters erst später beim Leernen auf. 
Aktuelle Methoden sind für alle Beteiligten nicht benutzerfreundlich genug.

Eine E-Mail zu schreiben ist für die meisten Schüler zeitaufwendig und auch nicht notwendig, wenn es nur um eine Bewertung in einem 1-5 Notenschema handelt. 
Für den Lehrenden, der die E-Mail erhaltet ist es auch schwierig zuzuordnen, 
zu welcher Unterrichtseinheit oder Thema die Rückmeldung sich bezieht. Zudem fehlt den Lehrenden die Möglichkeit, auf eine einfache 
Art und Weise, nachträglich um Feedback zu bitten.

\section{Zielsetzung}
\setauthor{Stefano Pyringer}
Unsere Ziel ist es, den Schülern und Lehrenden eine einfache Möglichkeit für eine schnelle einfache Bewertung zu bieten.
Folgende Anforderungen sollte die Anwendung erfülllen:

\textit{Schüler/Lehrerverwaltung}
\begin{itemize}
    \item Anlegen und Bearbeitung der Benutzerkonten
    \item Übersicht über die Schüler
\end{itemize}

\textit{Verwaltung der Feedback}
\begin{itemize}
    \item Vorlesungs- und Lehrerbewertungssystem
    \item Schüler- und Klassenbewertungssystem
    \item Kommentarmöglichkeit
    \item Statistik in Form einer PDF teilen oder ausdrucken
    \item Login mit der Schul-E-Mail
    \item Überblick und Vergleich die Feedbacks
\end{itemize}

\subsection{Geplantes Ergebnis}
Unser Ziel ist es, dass mindestens 70\% der Schüler der HTL-Leonding die Feedback-App benutzen werden. 
Es soll zu einer sichtbaren Verbesserung der Unterrichtsgestaltung kommen und auch damit zu weniger Missverständnissen 
und Konfikten zwischen Lehrer und Schülern kommen. Die Motiviation der Schüler wird durch die verbesserten Lehrbedinungen gesteigert und 
somit auch deren Lernerfolg.
Diese Ziele definieren einen guten Erfolg dieses Projektes.

\section{Theoretischer Hintergrund}
\setauthor{Stefano Pyringer}
\cite{BWLHTLbook}
\cite{FeedbackBWissen}

\subsection{Was ist Feedback?}

Ein Feedback ist eine offene Rückmeldung an eine Person oder Gruppe, wie ihr Verhalten von anderen 
wahrgenommen oder gedeutet wird. Diese Rückmeldung dient dazu, voneinander zu lernen. Denn Menschen lernen 
besonders gut, wenn sie die Ergebnisse ihrer Handlung sehen. Das Feedback macht diese Ergebnisse transparenter. 
Zudem hilft das Feedback, die Beziehungen zwischen den Personen zu klären, um die andere Person besser zu verstehen.

Rückmeldungen korrigieren die Verhaltensweisen, die der betroffenen Person und der Gruppe nicht weiterhelfen oder 
die der eigentlichen Intention nicht genügend angepasst oder konform sind. Erst nach Erhalt dieser Informationen ist der 
Feedback-Empfänger in der Lage, wie er sein Verhalten verändern könnte, damit das von ihm beabsichtige Ergebnis eintritt. 
Wichtig dabei ist, dass nur die betroffene Person über die Verhaltensänderung entscheidet und nicht der Feedbackgeber. 
Feedbacks werden auch dazu verwendet, positive Ergebnisse mitzuteilen, was dem Empfänger bestätigt, dass sein Handeln richtig ist.
Im Falle eines Fehlverhaltens wird das Feedback-Gespräch als Kritikgespräch bezeichnet.

Beim Feedback geben sollte man sich an diese Empfehlungen halten, damit eine Rückmeldung wirksam und lehrreich ist:

\begin{itemize}
    \item zuerst die positiven, dann die negativen Beobachtungen besprechen und mit einer positiven Rückmeldung enden (Sandwich-Feedback)
    \item Konkrete Situationen und Verhaltensweisen beschreiben
    \item Subjektive Satzformulierungen, z. B. "Ich hatte den Eindruck ..."
    \item Eigene Interpretationen und Verallgemeinerungen vermeiden
    \item Nach einer negativen Rückmeldung konstruktive Verbesserungsvorschläge machen
    \item Das Feedback möglichst zeitnah geben
\end{itemize}

Der Feedback-Empfänger sollte beachten, dass:
\begin{itemize}
    \item Feedback eine Chance zur individuellen Weiterentwicklung ist.
    \item den Gesprächpartner ausreden lässt.
    \item man zu seinen Schwächen stehen soll.
    \item Missverständnisse durch Nachfragen vermieden werden.
    \item man das Feedback auf sich wirken lassen sollte, bevor man sich dazu äußert oder Stellung nimmt.
    \item man nach dem Feedback darüber nachdenken sollte, welche Wünsche der anderen angenommen werden sollte und welche nicht.
\end{itemize}

\subsection{Johari-Fenster}
Die Persönlichkeitsmerkmale können unterschiedlich von Menschen oder von einem selbst wahrgenommen werden.
Das Johari-Fenster stellt die Unterschiede zwischen Selbst- und Fremdwahrnehmung dar. 
Es wurde 1955 von den amerikanischen Sozialpsychologen Joseph Luft und Harry Ingham entwickelt. Mithilfe des 
Feedback soll der öffentliche Bereich der Person erweitert werden.

\begin{figure}[h]
    \begin{center}
        \includegraphics*[width=8cm]{./pics/Johari.png}
        \caption[Johari-Fenster]{Johari-Fenster \cite{Johari}}
    \end{center}
\end{figure}

\subsubsection{Öffentliche Person}
In diesem Feld versammeln sich Merkmale, die der Person selbst und anderen Personen bekannt sind. 
Das Handeln der Person ist frei und unbeeinträchtigt von Ängsten und Vorbehalten.

\subsubsection{Offenkundige Person - blinder Fleck}
Hier finden sich Persönlichkeitsmerkmale, die der eigenen Person kaum bewusst ist, jedoch von anderen deutlich 
wahrgenommen werden können. Diese Merkmale sind sichtbar durch unbewusste Gewohnheiten und Verhaltensweisen, Vorurteile und 
persönliche Zu- und Abneigungen.

\subsubsection{Verborgene Person}
Diese Persönlichkeitsmerkmale werden von der Person bewusst verborgen sind nur dieser bekannt. Dies können 
heimliche Wünsche oder wunde Punkte sein.

\subsubsection{Unbewusste Person}
In diesem Feld sind die Merkmale der Person sich selbst und anderen Personen verborgen. Beispiel dafür sind 
verborgene Talente oder ungenützte Begabungen.

\section{Geplantes Ergebnis}
\setauthor{Stefano Pyringer}


\section{Aufbau der Diplomarbeit}
\setauthor{Stefano Pyringer}

\subsubsection{2. Systemarchitektur}
Als erstes wird das Datenmodel, Struktur und Kommunikation der Applikation erklärt. 

\subsubsection{3. Technologien}
In diesem Kapitel werden die verwendeten Technologien erklärt.

\subsubsection{4. Entwicklungsumgebungen}
Hier werden die verwendeten Werkzeuge für die Entwicklung der Feedback App erklärt und 
wieso sie für dieses Projekt ausgewählt wurden.

\subsubsection{5. Produktübersicht}
In der Produktübersicht werden die einzelnen Views der Benutzeroberfläche der mobilen Anwendung präsentiert 
und ihre Funktionen erklärt.

\subsubsection{6. Umsetzung}
In diesem Kaptitel wird die Umsetzung dargelegt, wie die Diplomarbeit realisiert wurde.

\subsubsection{7. Nicht realisierte Funktionen}
Zum Abschluss werden die nicht umgesetzten Funktionen der App kurz beschrieben.