\section{Visual Studio 2022}
\author{Mirzet sakonjic}
\cite{VS2022}
\begin{figure}[h]
    \begin{center}
        \includegraphics*[width=6cm]{pics/Visual-Studio-Logo.png}
        \caption[VS 2022 Logo]{Visual Studio 2022 Logo \cite{VS2022logo}}
    \end{center}
\end{figure}
\subsection*{Erklärung}
Visual Studio Enterprise 2022 ist eine integrierte Entwicklungsumgebung für verschiedene Hochsprachen von Microsoft welche im Jahr 2022 (Juni) veröffentlicht wurde.
Visual Studio ermöglicht es Programmierern, sowohl Win32/Win64-Programme als
auch Anwendungen für das .NET Framework zu entwickeln. Darüber hinaus lassen sich
mit Visual Studio Windows-Apps, dynamische Webseiten bzw. Webservices für das
Internet/Intranet oder Azure-Services entwickeln.
\subsection*{Funktionalität}
Visual Studio ist eine sehr umfangreiche und komfortable Entwicklungsumgebung. Sie
lässt sich gezielt auf die Anforderungen von Projekten anpassen. Mit dem VS-Installer
können zusätzliche Hochsprachen installiert oder deinstalliert werden.
Neben der Erweiterbarkeit stellt Visual Studio einen integrierten Debugger zur Verfügung. Dieser enthält die Funktion „Bearbeiten und Fortfahren“ und erlaubt das
nachträgliche Anhängen an bereits laufende Prozesse, sowohl am lokalen Rechner als
auch über das Netzwerk. Neben dem Debugger wird der Softwareentwickler durch eine
gute IntelliSense unterstützt.
\subsection*{Begründung und Verwendung}
Visual Studio ist die etablierteste Entwicklungsumgebung auf dem Markt, um .NET
zu programmieren, da es sich durch seinen Umfang und die gute Bedienbarkeit ausgezeichnet. Aufgrund der vielen Funktion, der Marktposition und der Tatsache, dass
sich Visual Studio bereits in vergangenen Projekten bewährt hat, wurde es für unsere
Arbeit gewählt. Die Enterprise Lizenz wurde von der HTL zur Verfügung gestellt und
darf ausschließlich für schulische Zwecke eingesetzt werden.
\newpage

\section{Visual Studio Code}
\author{unknown}
\lipsum[10-15]

\subsection{Thunder Client Extension}
\author{Stefano Pyringer}
\lipsum[5-10]

\subsection{ERD Editor Extension}
\author{Stefano Pyringer}
\lipsum[5-10]
\newpage

\section{Github}
\author{Mirzet Sakonjic}
\cite{GitHub}
\begin{figure}[h]
    \begin{center}
        \includegraphics*[width=8cm]{pics/GitHub_Logo.png}
        \caption[gitHub Logo]{GitHub Logo \cite{GithubLogo}}
    \end{center}
\end{figure}
\subsection*{Erklärung}
GitHub ist die primäre Plattform für Entwickler, um ihre Software zu hosten 
und zu verwalten. wurde 2008 gegründet und 2018 von Microsoft gekauft. 
Da GitHub dennoch zu einer wichtigen Anlaufstelle geworden ist, 
kann es in dieser Weise lange Zeit als soziales Netzwerk für Entwickler 
verstanden werden. Jeder hat dort sein eigenes Profil, einige arbeiten 
nur an Open-Source-Software und werden von Sendeanstalten finanziert. 
Andere führen ihre privaten Projekte nach der Arbeit durch oder unterstützen 
größere Projekte nur zum Spaß.
\subsection*{Funktionalität}
Um ein Programm, eine Website oder ähnliches bearbeiten oder durchsuchen zu können, 
muss das Repository oder Verzeichnis öffentlich sein oder Sie müssen dazu eingeladen 
werden. Und wenn dies nicht der Fall ist, kann nur der Ersteller des Repositorys 
daran arbeiten. An einem Verzeichnis können beliebig viele Entwickler arbeiten. 
Um einen Beitrag leisten zu können, müssen Sie das Repository forken, was bedeutet, 
dass Sie eine Kopie auf Ihrem eigenen Konto mit den aktuellen Daten erstellen. 
Wenn der Ersteller des ursprünglichen Repositorys dies später zulässt, können die 
beiden Repositorys wieder zusammengeführt werden. Dies wird als Zusammenführen 
bezeichnet. Sie können auch einfach dem Repository oder Entwickler folgen. 
GitHub wird von Microsoft betreuet.
\subsection*{Begründung und Verwendung}
GitHub ist die seriöseste und bekannteste Plattform auf dem Markt für Teamprojekte 
und -arbeiten, da sie sich durch Geschwindigkeit und Benutzerfreundlichkeit 
auszeichnet. Es wurde aufgrund seiner zahlreichen Funktionen und der Kompatibilität 
mit Visual Studio für unsere Arbeit ausgewählt. Es gibt keine Lizenz und es kann 
kostenlos für jeden Zweck verwendet werden.
\newpage

\section{Testen auf dem Endgerät}
\subsection{Android Emulator}
\subsection{iOS Emulator}