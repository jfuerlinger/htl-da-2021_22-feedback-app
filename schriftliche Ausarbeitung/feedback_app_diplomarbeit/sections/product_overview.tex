Die Produktübersicht zeigt die einzelnen Ansichten der Website und erläutert deren 
Funktionen. Die mobile Version wird für jedes Display (Android oder iOS) verglichen, 
um zu zeigen, dass die Feedback-Anwendung auf allen derzeit gängigen Endgeräten 
funktionsfähig und nutzbar ist. Für die mobile Version wurde das iPhone 12 Pro 
ausgewählt, da es sich um ein Smartphone mit sehr großem Marktanteil handelt.
\newpage

\section{Login}
\begin{figure}[h]
    \begin{center}
        \includegraphics*[width=5cm]{pics/LoginPage.png}
        \caption[Login]{Login Mobile Ansicht}
    \end{center}
\end{figure}
Durch den Aufruf der Anwendung gelangt der Nutzer auf die Anmeldeseite 
von Feedback. Die Anmeldeseite ist schwarz-weiß, schlicht und übersichtlich gestaltet.
Im Login Menü E-Mail-Adresse und Passwort eingegeben, welche in 
der Datenbank hinterlegt sind. Button "Anmelden" überprüft unsere 
eingegebenen Werte, und wenn sie korrekt sind, gelangen wir auf die Startseite.
Falls wir noch keine Daten eingegeben und ein Konto zum Anmelden erstellt haben, 
finden Sie unten die Schaltfläche zum Registrieren neuer Benutzer.
\newpage

\section{Startseite}
\begin{figure}[h]
    \begin{center}
        \includegraphics*[width=5cm]{pics/Startseite Schüler.png}
        \caption[Login]{Login Mobile Ansicht}
    \end{center}
\end{figure}

\subsection{Schüler}
\subsection{Lehrer}
\newpage

\section{Benutzerkontoverwaltung}
\begin{figure}[h]
    \begin{center}
        \includegraphics*[width=5cm]{pics/My Account.png}
        \caption[Benutzerkontoverwaltung]{Benutzerkontoverwaltung}
    \end{center}
\end{figure}
Im Profil kann der Benutzer seine gespeicherten Daten wie Passwort, Vorname oder
Schule aktualisieren.

\section{Feedback-Einheit}
\subsection{Feedback-Einheit erstellen}
\subsection{Feedback geben}

\section{Statistiken}
