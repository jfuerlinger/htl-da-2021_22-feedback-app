\section{.NET 6}
\setauthor{Stefano Pyringer}
\lipsum[5-12]

\section{ASP .NET Core}
\setauthor{Stefano Pyringer}
\lipsum[12-17]

\subsection{Identity}
\setauthor{Stefano Pyringer}
\lipsum[1-2]

\section{Entity Framework Core}
\setauthor{Stefano Pyringer}
\lipsum[17-20]

\subsection{MSSQLLocalDB}
\setauthor{Stefano Pyringer}
\lipsum[30-40]

\section{Open API Swagger}
\setauthor{Stefano Pyringer}
\lipsum[20-22]

\section{JWT Token}
\setauthor{Stefano Pyringer}
\lipsum[22-25]
\newpage

\section{Xamarin}
\setauthor{Mirzet Sakonjic}
\cite{XML}
\begin{figure}[h]
    \begin{center}
        \includegraphics*[width=8cm]{pics/Xamarin_logo.png}
        \caption[Xamarin Logo]{Xamarin Logo \cite{XMLlogo}}
    \end{center}
\end{figure}
\subsection*{Erklärung}
Xamarin ist eine Open-Source-Plattform zum Aufbau moderner 
und leistungsfähiger Applikationen für das Betriebssystem Apples (iOs), 
Android und Windows mit .NET. Xamarin ist eine Abstraktionsschicht, 
die die Verständigung von freigegebenem Code mit dem zugrunde liegenden 
Plattformcode verwaltet. Xamarin wird in einer verwalteten Umgebung 
ausgeführt, die Annehmlichkeiten wie Speicherzuweisung und Garbage 
Collection offeriert.
\subsection*{Funktionalität}
Xamarin realisiert es Entwicklern, durchschnittlich 90 Prozent ihrer App 
plattformübergreifend miteinander zu nutzen. Dieses Muster erlaubt es 
Entwicklern, ihre restlose Geschäftslogik in einer einzigen Sprache 
zu schreiben (oder vorhandenen Anwendungscode wiederzuverwenden), 
allerdings auf jedweder Plattform native Leistung, Look und Verhalten 
zu erzielen.
\subsection*{Begründung und Verwendung}
Xamarin ist für Entwickler mit den folgenden Zielen:
\begin{enumerate}
    \item Code, Test und Geschäftslogik plattformübergreifend teilen
    \item Plattformübergreifende Anwendungen in C-Sharp mit Visual Studio schreiben
\end{enumerate}
Es wurde für unsere Arbeit ausgewählt, weil es den zusätzlichen Tools 
der Visual Studio-Software durchaus nahe kommt und zudem für uns kostenlos 
nutzbar ist. Die Enterprise-Lizenz wird von der HTL bereitgestellt und 
darf nur für schulische Zwecke verwendet werden.