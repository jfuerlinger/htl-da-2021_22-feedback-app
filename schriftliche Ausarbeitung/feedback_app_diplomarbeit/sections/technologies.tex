\section{.NET 6}
\setauthor{Stefano Pyringer}
Das .NET wird für die Entwicklung und Ausführung von Anwendungsprogrammen verwendet und ist Teil der .NET-Plattform.
Sie ist mit einer MIT-Lizenz lizensiert und ist somit eine freie open-source Software. Das .NET-Framework wurde von Microsoft 
2016 unter den Namen .NET Core modernisiert. .NET ist der Nachfolger von .NET Core seit Dezember 2020 und ist in der Version 6.0.7 verfügbar (Stand Juli 2022).

\subsection{Allgemeine Eigenschaften}
Microsoft die Konzeptideen von Java, aufgrund des hohen Erfolgs der Sprache,
übernommen und versucht, die bekannten Schwachstellen der Sprache auszumerzen und dadurch die Messlatte mit 
.NET spürbar höher gelegt.

% ev. Bilder einfügen
Folgenede Eigenschaften bringt .NET mit sich:

\subsubsection{Objektorientierung}
.NET ist komplett objektbasiert. Auch einfache Datentypen wie Integer werden als Objekte behandelt.

\subsubsection{WinAPI-32-Ersatz}
Microsoft will langfristig die Win32-API durch Klassen des .NET-Frameworks ersetzten. Alle Sprachen greifen 
daher auf die gleiche Bibliothek zurück. Dadurch verwischen die charackterischtischen Merkmale der verschiedenen Sprachen. 
Die Wahl einer Sprache im .NET ist nicht mehr mit den Entscheidungen gleichzusetzen, wie effizient eine Anwendung geschrieben werden kann 
oder was sie zu leisten imstande ist.

\subsubsection{Platttformunabhängigkeit}
Auf .NET bassierenden Anwendungen laufen in einer Umgebung, die mit der virtuellen Maschine von Java verglichen werden kann.
Erst zur Laufzeit der Anwendung wird der Maschinencode generiert. Aufgrund der öffentlichen Dokumentation der Spezifikation von 
der Common Language Runtime (CLR) lässt sich sie sich auf Plattformen portieren wie Unix oder Linux. 
Quelloffene Beispiele sind Mono oder CoreCLR.

\subsection{Sprachunabhängigkeit}
Eine in C\# geschriebene Klasse kann aus jeder anderen .NET-kompitablen Sprache wie F\# aufgerufen werden, ohne die Verwendung über eine 
spezifizierte Schnittstellentechnologie wie COM/COM+ gehen zu müssen. Beispielweise lässt sich eine C\# implementierte Klasse aus einer 
VB.NET-KLasse ableiten und auch umgekehrt ist dies möglich.

\subsection{Architektur und Anwedungsgebiete}
Das Software-Development-Kit (SDK) unterstützt Windows (ab 7, 32/64 Bit und Arm), macOS (ab 10.12), Linux-Distrubtionen(64 Bit, Arm) 
und für Docker oder Snappy existieren offizelle Images. .NET besteht aus 2 Hauptkomponenten CoreCLR und CoreFX. 
Sie sind vergleichbar mit der Common Language Runtime (CLR) und der  Framework Class Library (FCL) 
von .NET Framework's Common Language Infrastructure (CLI). .NET unterstüzt 

Folgende Entwicklungsumgebungen werden von der SDK unterstützt:
\begin{itemize}
    \item Visual Studio (ab 2022 auch für macOS)
    \item Visual Studio Code
    \item per Kommandozeile mit dem .NET SDK
    \item JetBrains Rider 
\end{itemize}

Standardmäßig können für die Entwicklung von .NET Apps die Programmiersprachen C\#, F\# oder Visual Basic verwendet werden. 
.NET kann in der Funktionalität mithilfe von NuGet-Packages erweitert werden. Folgende Anwendungsgebiete unterstützt .NET 6:
\begin{itemize}
    \item Web-Apps und Micro-Services (ASP.NET Core)
    \item Kommandozeilen Programme
    \item Klassenbibliotheken
    \item GUI-Applikationen für Windows (UWP, WPF) und Cross-Plattform-Apps (.NET MAUI, Xamarin)
    \item Machine Learning (ML.NET, Apache Spark for .NET)
    \item Game Development (Unity, Cryengine, MonoGame, etc.)
    \item Internet of Things
\end{itemize}

\section{ASP.NET Core}
\setauthor{Stefano Pyringer}
ASP.NET Core ist ein modulares Open-Source Web-Framework für die Entwicklung von modernen Web-Anwendungen.
Es wurde von Microsoft entwickelt und ist der Nachfolger von ASP.NET seit .NET 5. Das Web-Framework ist ein Bestandteil von 
.NET. ASP.NET Core wurde von Grund auf neu entwickelt und unterscheidet sich wesentlich von Vorgänger Versionen. 

Die wichtigsten Vorteile zu ASP.NET sind:
\begin{itemize}
    \item Entwicklung von Web-UI und Web-APIs in einer einheitlichen Umgebung
    \item Open-Source und Community freundlich
    \item Plattformunabhängigkeit (Windows, MacOS, Linux)
    \item Moderne Entwicklungstools
    \item Leichtgewichtige, modulare und leistungsstarke HTTP-Request Pipeline
    \item Umfassende Testmöglichkeiten
    \item Dependency Injection
    \item Razor Pages und Blazor ermöglichen dynamische Webseiten mit .NET Programmiersprachen
    \item Cloud kompitabel
    \item Mehrere Hosting Möglichkeiten:
    \begin{itemize}
        \item Kestrel
        \item IIS
        \item HTTP.sys
        \item Nginx
        \item Apache
        \item Docker
    \end{itemize}
\end{itemize}

\subsection{Anwendungsgebiete}
\setauthor{Stefano Pyringer}

\subsubsection{Interaktive clientseitige Web-Apps mit Blazor/Razor View Engine}
Mithilfe von Blazor ist das Erstellen von clientseitigen Webbenutzeroberflächen möglich.
Zudem bietet die Razor View Engine die Möglichkeit, .NET-Sprachblöcke in HTML-Seiten einzubetten.
Die UI wird anschließend als HTML und CSS gerendert inklusive umfassender Browserunterstützung (auch mobile Browser).
Mit der Technik des Client-Side-Renderings werden die Skripte im Browser ausgeführt und verarbeitet.

\subsubsection{Web-APIs}
Mit ASP.NET Core können RESTful-basierte Webservices (HTTP-API) entwickelt werden. Die Services werden von Controllern gesteuert und diese 
unterstützen CRUD-Datenoperationen (Create, Read, Update, Delete).
Die Schnittstellen können mit Swagger/Open-API dokumentiert werden.

\subsubsection{Web-Apps und APIs mit MVC}
ASP.NET Core MVC ist ein zusätzliches umfassendes Framework bassierend auf dem Model-View-Controller-Entwursfmusters. Damit werden 
die einzelnen Bereiche getrennt, was die Entwicklung, Organisition und Testbarkeit erleichtert. 
ASP.NET Core MVC wird parallel zu Razor Pages als Alternative weiterhin unterstützt.

\subsubsection{Web-Apps und APIs mit Razor Pages}
Razor Pages ist der Nachfolger von MVC in ASP.NET Core und ersetzt ASP.NET Web Forms. 
Sie bassiert auf dem MVC-Framework, jedoch ist die Komplexität reduziert. Razor Pages verwendet das 
MVVM-Entwurfsmuster (Model-View-View-Model). Somit entfällt der Controller.


\subsubsection{Echtzeit-Web-Apps mithilfe von SignalR}
SignalR ist eine Open-Source-Bibliothek mithilfe die Hinzuufügung von Echtzeitwebfunktionen zu Apps zu vereinfachen.
Diese Funktionen ermöglichen serverseitigen Code, Inhalte sofort an die Clients zu senden. SignalR verarbeitet die Verbindungsveraltung 
automatisch.

SignalR ist besonders für Szenarien geeignet, wo man Daten vom Server mit einer hohen Aktualisierungsfrequenz benötigt. 
Solche Kanditen können sein:
\begin{itemize}
    \item Dashboard oder Überwachungsdienste wie Sofortupdates wie Reisehinweise oder Börsenwerte.
    \item Soziale Netzwerke-Apps mit Chatroom-Funktion
    \item Applikationen, die wiederholt in kürzester Zeit Daten benötigen (z. B.Gaming, GPS, Voting, Auktionen, etc.)
\end{itemize}

\subsubsection{gRPC-Dienste}
RPC steht für Remote Procedure Call und eignet sich dafür, Mikroservices miteinander kommunizieren zu lassen.
Das dRPC-Framwork ist sprachunabhängig und hochleistungsfähig.

Es eignet sich besonders für folgende Szenarien:
\begin{itemize}
    \item Simple Mikroservices, wo Effizienz wichtig ist
    \item Mehrsprachige Systeme
    \item Point-to-Point-Dienste, die in Echtzeit Streaminganforderungen oder -antworten verarbeiten müssen.
\end{itemize}

\subsubsection{Datengesteuerte Web-Apps}

\subsubsection{Hybrid-Apps}
Die Technologien können miteinander kombiniert werden. So ist eine 

\subsection{Identity}
\setauthor{Stefano Pyringer}


\section{Entity Framework Core}
\setauthor{Stefano Pyringer}


\subsection{MSSQLLocalDB}
\setauthor{Stefano Pyringer}


\section{Open API Swagger}
\setauthor{Stefano Pyringer}


\section{JWT Token}
\setauthor{Stefano Pyringer}

\newpage

\section{Xamarin}
\setauthor{Mirzet Sakonjic}
\cite{XML}
\begin{figure}[h]
    \begin{center}
        \includegraphics*[width=9cm]{pics/Xamarin_logo.png}
        \caption[Xamarin Logo]{Xamarin Logo \cite{XMLlogo}}
    \end{center}
\end{figure}
\subsection*{Erklärung}
Im Jahr 2011 starteten die ehemaligen Mono-Entwickler das Unternehmen 
namens Xamarin, um auf dem aufkommenden Markt der mobilen Betriebssysteme 
eine plattformübergreifende Entwicklungsumgebung zu schaffen. 
Mono ist eine weitere, quelloffene Implementierung von Microsofts .NET Framework. 
Sie erlaubt die Änderung von plattformunabhängiger Software auf den Direktiven 
der Common Language Infrastructure und der Programmiersprache C\#.
Das gleichnamige, ein Jahr später vorgestellte Produkt ermöglichte den 
Einsatz der Programmiersprache C-Sharp zur Entwicklung von 
plattformübergreifenden Anwendungen für Windows, aber auch für MacOS. 
Ab der Version 2.0, die 2013 vorgestellt wurde, kamen als Zielplattformen 
die mobilen Betriebssysteme iOS und Android dazu, so dass nun aus Visual 
Studio heraus Programme für MacOS, iOS und Android entwickelt werden 
konnten. 
Xamarin ist eine Open-Source-Plattform zum Aufbau moderner 
und leistungsfähiger Applikationen für das Betriebssystem Apples (iOs), 
Android und Windows mit .NET. Xamarin ist eine Abstraktionsschicht, 
die die Verständigung von freigegebenem Code mit dem zugrunde liegenden 
Plattformcode verwaltet. Xamarin wird in einer verwalteten Umgebung 
ausgeführt, die Annehmlichkeiten wie Speicherzuweisung und Garbage 
Collection offeriert.
\subsection*{Funktionalität}
Xamarin realisiert es Entwicklern, durchschnittlich 90 Prozent ihrer App 
plattformübergreifend miteinander zu nutzen. Dieses Muster erlaubt es 
Entwicklern, ihre restlose Geschäftslogik in einer einzigen Sprache 
zu schreiben (oder vorhandenen Anwendungscode wiederzuverwenden), 
allerdings auf jedweder Plattform native Leistung, Look und Verhalten 
zu erzielen.
\subsection*{Begründung und Verwendung}
Xamarin ist für Entwickler mit den folgenden Zielen:
\begin{enumerate}
    \item Code, Test und Geschäftslogik plattformübergreifend teilen
    \item Plattformübergreifende Anwendungen in C\# mit Visual Studio schreiben
\end{enumerate}
Es wurde für unsere Arbeit ausgewählt, weil es den zusätzlichen Tools 
der Visual Studio-Software durchaus nahe kommt und zudem für uns kostenlos 
nutzbar ist. Die Enterprise-Lizenz wird von der HTL bereitgestellt und 
darf nur für schulische Zwecke verwendet werden.