\section{.NET 6}
\setauthor{Stefano Pyringer}
Das .NET wird für die Entwicklung und Ausführung von Anwendungsprogrammen verwendet und ist Teil der .NET-Plattform.
Sie ist mit einer MIT-Lizenz lizensiert und ist somit eine freie open-source Software. Das .NET-Framework wurde von Microsoft 
2016 unter den Namen .NET Core modernisiert. .NET ist der Nachfolger von .NET Core seit Dezember 2020 und ist in der Version 6.0.7 verfügbar (Stand Juli 2022).

\subsection{Architektur und Anwedungsgebiete}
Das Software-Development-Kit (SDK) unterstützt Windows (ab 7, 32/64 Bit und Arm), macOS (ab 10.12), Linux-Distrubtionen(64 Bit, Arm) 
und für Docker oder Snappy existieren offizelle Images. .NET besteht aus 2 Hauptkomponenten CoreCLR und CoreFX. 
Sie sind vergleichbar mit der Common Language Runtime (CLR) und der  Framework Class Library (FCL) 
von .NET Framework's Common Language Infrastructure (CLI). .NET unterstüzt 

Folgende Entwicklungsumgebungen werden von der SDK unterstützt:
\begin{itemize}
    \item Visual Studio (ab 2022 auch für macOS)
    \item Visual Studio Code
    \item per Kommandozeile mit dem .NET SDK
    \item JetBrains Rider 
\end{itemize}

Standardmäßig können für die Entwicklung von .NET Apps die Programmiersprachen C\#, F\# oder Visual Basic verwendet werden. 
.NET kann in der Funktionalität mithilfe von NuGet-Packages erweitert werden. Folgende Anwendungsgebiete unterstützt .NET 6:
\begin{itemize}
    \item Web-Apps und Micro-Services (ASP.NET Core)
    \item Kommandozeilen Programme
    \item Klassenbibliotheken
    \item GUI-Applikationen für Windows (UWP, WPF) und Cross-Plattform-Apps (.NET MAUI, Xamarin)
    \item Machine Learning (ML.NET, Apache Spark for .NET)
    \item Game Development (Unity, Cryengine, MonoGame, etc.)
    \item Internet of Things
\end{itemize}

\section{ASP .NET Core}
\setauthor{Stefano Pyringer}


\subsection{Identity}
\setauthor{Stefano Pyringer}


\section{Entity Framework Core}
\setauthor{Stefano Pyringer}


\subsection{MSSQLLocalDB}
\setauthor{Stefano Pyringer}


\section{Open API Swagger}
\setauthor{Stefano Pyringer}


\section{JWT Token}
\setauthor{Stefano Pyringer}

\newpage

\section{Xamarin}
\setauthor{Mirzet Sakonjic}
\cite{XML}
\begin{figure}[h]
    \begin{center}
        \includegraphics*[width=8cm]{pics/Xamarin_logo.png}
        \caption[Xamarin Logo]{Xamarin Logo \cite{XMLlogo}}
    \end{center}
\end{figure}
\subsection*{Erklärung}
Xamarin ist eine Open-Source-Plattform zum Aufbau moderner 
und leistungsfähiger Applikationen für das Betriebssystem Apples (iOs), 
Android und Windows mit .NET. Xamarin ist eine Abstraktionsschicht, 
die die Verständigung von freigegebenem Code mit dem zugrunde liegenden 
Plattformcode verwaltet. Xamarin wird in einer verwalteten Umgebung 
ausgeführt, die Annehmlichkeiten wie Speicherzuweisung und Garbage 
Collection offeriert.
\subsection*{Funktionalität}
Xamarin realisiert es Entwicklern, durchschnittlich 90 Prozent ihrer App 
plattformübergreifend miteinander zu nutzen. Dieses Muster erlaubt es 
Entwicklern, ihre restlose Geschäftslogik in einer einzigen Sprache 
zu schreiben (oder vorhandenen Anwendungscode wiederzuverwenden), 
allerdings auf jedweder Plattform native Leistung, Look und Verhalten 
zu erzielen.
\subsection*{Begründung und Verwendung}
Xamarin ist für Entwickler mit den folgenden Zielen:
\begin{enumerate}
    \item Code, Test und Geschäftslogik plattformübergreifend teilen
    \item Plattformübergreifende Anwendungen in C-Sharp mit Visual Studio schreiben
\end{enumerate}
Es wurde für unsere Arbeit ausgewählt, weil es den zusätzlichen Tools 
der Visual Studio-Software durchaus nahe kommt und zudem für uns kostenlos 
nutzbar ist. Die Enterprise-Lizenz wird von der HTL bereitgestellt und 
darf nur für schulische Zwecke verwendet werden.