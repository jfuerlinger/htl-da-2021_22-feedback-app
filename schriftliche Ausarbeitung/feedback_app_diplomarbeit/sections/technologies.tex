\section{.NET 6}
\setauthor{Stefano Pyringer}
\lipsum[5-12]

\section{ASP .NET Core}
\setauthor{Stefano Pyringer}
\lipsum[12-17]

\subsection{Identity}
\setauthor{Stefano Pyringer}
\lipsum[1-2]

\section{Entity Framework Core}
\setauthor{Stefano Pyringer}
\lipsum[17-20]

\subsection{MSSQLLocalDB}
\setauthor{Stefano Pyringer}
\lipsum[30-40]

\section{Open API Swagger}
\setauthor{Stefano Pyringer}
\lipsum[20-22]

\section{JWT Token}
\setauthor{Stefano Pyringer}
\lipsum[22-25]

\section{Xamarin}
\setauthor{Mirzet Sakonjic}
\subsection*{Erklärung}
Im Jahr 2011 starteten die ehemaligen Mono-Entwickler 
das Unternehmen namens Xamarin, um auf dem aufkommenden 
Markt der mobilen Betriebssysteme eine plattformübergreifende
Entwicklungsumgebung zu schaffen. Das gleichnamige,
ein Jahr später vorgestellte Produkt ermöglichte
den Einsatz der Programmiersprache C-Sharp zur Entwicklung
von plattformübergreifenden Anwendungen für Windows, 
aber auch für MacOS. Ab der Version 2.0, die 2013 vorgestellt wurde, 
kamen als Zielplattformen die mobilen Betriebssysteme iOS und 
Android dazu, so dass nun aus Visual Studio heraus Programme für 
MacOS, iOS und Android entwickelt werden konnten.
\subsection*{Funktionalität}
\subsection*{Begründung und Verwendung}