\begin{spacing}{1}
    \chapter*{Abstract}
\end{spacing}
\begin{figure}[h]
  \includegraphics*[width=5cm]{./pics/feedback_stars.jpg}
\end{figure}

\subsubsection*{Inital position}
Feedback is a way for people and companies to improve themselves and their products. 
Especially in the age of the internet, evaluation is quick and easy. In education, it is rarely possible to give feedback, 
as teachers immediately move on to the next lesson.
Our goal is to provide students and teachers with an easy way to do a quick assessment.

\subsubsection{The Application}
The feedback app is based on .NET technologies. The data is stored on a server, 
which was developed with ASP.NET Core and enables external data access via an HTTP API. This makes an implementation of a Frontend very easy. 
This project focuses on a user interface for 
mobile operating systems Android and iOS, which was implemented with the help of Xamarin.

\subsubsection{Result}
Our application allows teachers to give feedback to students on their teaching units in a simple and quick way.

\newpage
\begin{spacing}{1}
    \chapter*{Zusammenfassung}
\end{spacing}
\begin{figure}[h]
  \includegraphics*[width=5cm]{./pics/feedback_stars.jpg}
\end{figure}

\subsubsection{Ausgangslage}
Feedback ist eine Möglichkeit für Personen und Unternehmen, sich und ihre Produkte und Dienstleistungen zu verbessern. 
Besonders im Zeitalter des Internets ist eine Bewertung einfach und schnell möglich. Im Bildungsbereich ist es selten möglich, ein sofortiges Feedback zu geben, 
da die Lehrenden sofort in die nächste Unterrichtseinheit wechseln.
Unsere Ziel ist es, den Schülern und Lehrenden eine einfache Möglichkeit für eine schnelle Bewertung zu bieten.

\subsubsection{Die Applikation}
Die Feedback App basiert auf .NET Technologien. Die Daten werden auf einem Server gespeichert, 
der mit ASP.NET Core entwickelt wurde und mittels einem HTTP-API den Datenzugriff nach außen ermöglicht, welches 
die Umsetzung eines Frontends sehr flexibel macht. Dieses Projekt konzentriert sich auf eine Benutzeroberfläche für die 
mobilen Betriebssysteme Android und iOS, die mithilfe von Xamarin umgesetzt wurde.

\subsubsection{Ergebnis}
Unsere Anwendung ermöglicht es Lehrenden den Schülern, auf eine einfache und schnelle Weise Feedback zu ihren Lehreinheiten zu geben.

\newpage
\subsubsection{Danksagung}
Wir möchten uns an dieser Stelle herzlich bei unserem Diplomarbeitsbetreuer Prof. Josef Fürlinger bedanken, 
dass er uns in allen Besprechungen stets mit seinem Fachwissen zur Seite gestanden ist und uns durch dieses Projekt geleitet hat.
Des Weiteren möchten wir uns auch noch bei unseren Familien und Freunden bedanken welche uns, während diesem Projekt immer unterstützt haben.

