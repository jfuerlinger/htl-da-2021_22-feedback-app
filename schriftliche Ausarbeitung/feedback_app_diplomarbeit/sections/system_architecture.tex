\section{Übersicht}
\setauthor{Stefano Pyringer}
\lipsum[10]

\section{Backend}
\setauthor{Stefano Pyringer}

\subsection{Datenstruktur}
\subsection{Models}
\subsection{API-Services}

\section{Frontend}
\setauthor{Mirzet Sankonjic}

\subsection{Models}
Jedes Model ist ein Interface. In diesem Interface wird ein neues Datenmodel definiert,
welches in allen Componenten und Services verwendet werden kann. Dieses Datenmodel
ist ident mit den Data Transfer Objects oder Entitäten des Servers und dient als
Schnittstelle.
\subsection{Services}
Die Daten werden vom Client über einen Service per API an den Server gesendet
und wieder empfangen. Hierfür wird das HttpClientModule eingebunden. Es stehen
die HTTP-Requests GET, POST, PUT, PATCH und DELETE zur Verfügung. Die
Services werden dann in den benötigten Componenten injiziert im Konstruktor. Durch
die typisierte Variante wird ein Observable <> zurückgegeben.