\section{Übersicht}
\setauthor{Stefano Pyringer}

Diese Abbildung bietet einen groben Überblick über die Technologien, die für die Erstellung und Betrieb der Feedback App verwendet wurden.
Für die Umsetzung der Applikation wurde die Programmiersprache C\# verwendet. Die Technologien werden in Kapitel 3 ausführlich erklärt.

\begin{figure} [h]
    \includegraphics*[width=15cm]{./pics/architecture_overview.png}
    \caption[overview architecture]{Systemarchitektur Überblick \cite{CSharpLogo} \cite{XamarinLogo}}
\end{figure}

\subsubsection{Backend}
\begin{itemize}
    \item .NET 6
    \item \begin{itemize}
        \item ASP.NET Core
        \item Entity Framework Core
        \item ASP.NET Core Identity
    \end{itemize}
\end{itemize}

Für das Backend wurde die Software-Entwicklungsplattform .NET 6 verwendet, um Datenstruktur, Repositories und Authentifizierung zu realisieren. 
Es wird mittels HTTP-API mit dem Frontend kommuniziert.

\subsubsection{Frontend}
Das Frontend handhabt die Interaktionen des Benutzers und stellt die von der API zur Verfügung gestellten Daten dar. Der Client ist auf mobilen Endgeräten 
mit dem Betriebssystem Android und iOS lauffähig. Das Frontend wurde mit Xamarin realisiert.

\section{Backend}
\setauthor{Stefano Pyringer}

\subsection{Datenstruktur}
\subsection{Models}
\subsection{API-Services}

\section{Frontend}
\setauthor{Mirzet Sankonjic}

\subsection{Models}
Jedes Model ist ein Interface. In diesem Interface wird ein neues Datenmodel definiert,
welches in allen Componenten und Services verwendet werden kann. Dieses Datenmodel
ist ident mit den Data Transfer Objects oder Entitäten des Servers und dient als
Schnittstelle.
\subsection{Services}
Die Daten werden vom Client über einen Service per API an den Server gesendet
und wieder empfangen. Hierfür wird das HttpClientModule eingebunden. Es stehen
die HTTP-Requests GET, POST, PUT, PATCH und DELETE zur Verfügung. Die
Services werden dann in den benötigten Componenten injiziert im Konstruktor. Durch
die typisierte Variante wird ein Observable <> zurückgegeben.